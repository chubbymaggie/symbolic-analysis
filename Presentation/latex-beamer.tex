%\documentclass{beamer}
\documentclass[mathserif,10pt]{beamer}

\usepackage{beamerthemesplit}
\usepackage{graphics}
\usepackage{epsfig}
\usepackage{algorithm}
\usepackage{verbatim}
\usepackage{listings}
\usepackage{framed}
\usepackage{pstricks}
\usepackage{pst-node,pst-tree}
\usepackage{pst-rel-points}
\usepackage{flexiprogram}
\usepackage[UKenglish]{babel}
\usepackage{hyperref}
\usepackage{pst-coil}
\usepackage{color}
\usepackage{epsfig}
%\usepackage{tikz}
%\usepackage{multirow}

\usefonttheme{serif}

\newcommand{\cmt}[1]{}
%\noindent

\setcounter{tocdepth}{1}
\lstset{language=[ANSI]C}
\lstset{% general command to set parameter(s)
  basicstyle=\footnotesize\tt, % print whole listing small
    identifierstyle=, % nothing happens
    commentstyle=\color{red}, % white comments
    showstringspaces=false, % no special string spaces
    lineskip=1pt,
    captionpos=b,
    frame=single,
    breaklines=true
      %\insertauthor[width={3cm},center,respectlinebreaks]
}


\setbeamercovered{transparent=0}

\lstset{classoffset=0,
  morekeywords={},keywordstyle=\color{black},
  classoffset=1,
  classoffset=0}% restore default

  \usetheme{CambridgeUS}
  \usecolortheme{dolphin}

  \title[Static Analysis Debugging with Symbolic Execution]{Static Analysis Debugging with Symbolic Execution}
  \author[]{{\textbf{Theodoros Kasampalis, Sandeep Dasgupta}} }
  \begin{document}

  \begin{frame}
  \titlepage
  \end{frame}
  \usebeamertemplate{mytheme}

  \AtBeginSection[]
{
  \begin{frame}<beamer>
    \frametitle{Outline}
  \tableofcontents[currentsection]
    \end{frame}
}

\defverbatim[colored]\lstI{
\begin{lstlisting}[language=C++,basicstyle=\tiny,keywordstyle=\color{red}]
  1. int main() {
  2.   int x=1 , y=2;
  3.   int*  p = (int *)malloc(sizeof(int));

  4.   klee_make_symbolic(&x, sizeof(x), "x");
  5.   klee_make_symbolic(&y, sizeof(y), "y");
    /*
    ** If we skip to make y symbolic, then we may miss the 
    ** opportunity of catching a potential pointer analysis
    ** bug. For ex. what if the pointer analysis infers that
    ** *p and the heap object at line 7 mayNOT alias.
    */

        if(0 != x*y) {
  6.      p = (int *)malloc(4);
        } else {
          if(y == 0) {
  7.        p = (int *)malloc(4);
          }
        }
  8.    return *p;
  }

\end{lstlisting}
}

\defverbatim[colored]\lstII{
\begin{lstlisting}[language=C++,basicstyle=\tiny,keywordstyle=\color{red}]
  struct S {
    int member; 
  };
  struct S data[] =
  {
    { 1,2 },
    { 3,4 },
  };
  int main(int argc, char** argv) {

    int x= 0 ;
    struct S* z;
  
    klee_make_symbolic(&x, sizeof(x), "X");
    /*
    **  Without the following klee_assume, the dereference z->x  gets resolved to many
    **  spurious memory objects. 
    **  Generated in-bound constraints on the fly to prevent this.
    */
    klee_assume(x >= 0 &  x <= 1 );
  
    z   = &data[x];
    ... = z->member ; 

    return 0;
}
\end{lstlisting}
}

\defverbatim[colored]\lstIII{
\begin{lstlisting}[language=C++,basicstyle=\tiny,keywordstyle=\color{red}]
  /* The bug shows up when there is a must alias check between 
  ** x (at line 1) and the bitcast of x (at line 3).
  */
  int main(int argc, char **argv) {
    int *A[5];
    for (int i = 0; i < 5; ++i) {
      A[i] = (int*) malloc((i+1)*sizeof(int));
    }
 
    int *x, a;
    char *y;
        
    for (int i = 0; i < 5; ++i) {        
   1.   x = A[i];     
   2.   a = *x;      
   3.   y = (char *) x;
    }
    return *y;
  }
}

\end{lstlisting}
}

\defverbatim[colored]\algoCheckerIII{
\begin{lstlisting}[language=C,basicstyle=\tiny,keywordstyle=\color{red}]
  foreach(loadI = load instructions) {
    base_address  = `base address` of the loadI
    foreach( `pointer` in the same function scope as the load instruction) { 
      result = mustAlias_OR_mayNOTAlias(`base_address`, `pointer`) // Querying the alias analysis.
      if( result ==  must-alias) { 
        if (`base_pointer` and 'pointer' DO NOT point to the same run-time memory object) {
          error
        }
      }
      if (result ==  mayNot-alias) { 
        if (`base_pointer` and 'pointer' point to the same run-time memory object) {
          error
        }
      }
    }
  }
\end{lstlisting}
}

\section{Static Analysis}
\frame
{
  \frametitle{\secname}
  \begin{itemize}[<+->]
    \item Infer source code properties without execution
    \vspace{1cm}
    \item Examples:
    \begin{itemize}
      \item Pointer Analysis
      \item Liveness Analysis
    \end{itemize} 
    \vspace{1cm}
    \item Applications:
    \begin{itemize}
      \item Compilers
      \item Security
      \item Software Engineering
    \end{itemize} 
    \vspace{1cm}
    \item Inferred properties true for any execution
  \end{itemize} 
}

\section{Debugging a Static Analysis Implementation}
\frame
{
  \frametitle{\secname}
  \begin{itemize} [<+->]
    \item Semantic bugs
    \begin{itemize}
      \item no crash
      \item erroneous results
    \end{itemize} 
    \vspace{1cm}
    \item Effect visible in client code
    \begin{itemize}
      \item Hard to trace back
    \end{itemize} 
    \vspace{1cm}
    \item Static analysis specific tests
    \begin{itemize}
      \item small regression tests
    \end{itemize} 
  \end{itemize} 
}

\section{Related Work}
\frame
{
  \frametitle{\secname}
  \begin{itemize} [<+->]
    \item Compiler testing through miscompilation detection:
    \begin{itemize}
      \item Test suites (LLVM test suite)
      \item Randomly generated tests (Csmith, PLDI 11)
      \item Equivalence Modulo Inputs (Orion, PLDI 14)
    \end{itemize} 
    \vspace{1cm}
    \item Dynamic alias analysis error detection (NeonGoby, FSE 13)

    \vspace{1cm}
    \item Symbolic execution (KLEE, OSDI 08) 
    \item Concolic execution (zesti, ICSE 12 - SAGE, ICSE13)
  \end{itemize} 
}

\section{Our Idea}
\frame
{
  \frametitle{\secname}
  \begin{itemize} [<+->]
    \item Check inferred properties during symbolic execution
    \begin{itemize}
      \item Apply analysis to an input program
      \item Symbolically execute the input program
      \item Check whether inferred properties hold
    \end{itemize} 
    \vspace{1cm}
    \item Direct testing of static analysis code
    \vspace{1cm}
    \item Static analysis inferences checked thoroughly
    \begin{itemize}
      \item High path coverage of the input program
      \item Big input program size
    \end{itemize} 
  \end{itemize} 
}

\section{System Status Overview}
\frame
{
  \frametitle{\secname}
  \begin{itemize} [<+->]
    \item Implementation for checking an LLVM Alias Analysis (including tbaa, basicaa)
    \vspace{1cm}
    \item Checks incorporated within zesti
    \vspace{1cm}
    \item Checks on all pointer dereferences
    \vspace{1cm}
    \item Reachability analysis for inputs affecting pointer values [under implementation]
    \vspace{1cm}
    \item Testing with LLVM test suite programs
  \end{itemize} 
}

\section{Implementation}
\subsection{Symbolic Execution}
\frame
{
  \frametitle{\subsecname}
  \begin{itemize} 
    \item Symbolic execution using klee  
      \cmt{
        The symbolic analysis engine that we first explored was KLEE which serves a major
          benefits:
          for any illegal dreference, assertion failure or an exit , it tries to find out if there exists any  input value that has caused that error.
          klee does so by running the programs symbolically and while doing so generate the contraint that exactly describe the set of values
          possible on a given path. When klee founds an error, klee solves the
          path constraint and generate a set of input that will lead to the same error path.

          But the problem with klee is that it tries to hit every line
          of executable code in the program leading to huge runtime.  
      }
    \vspace{1cm} \item Migration from Klee to Zesti (a variant of klee) 
      \cmt{
        To mitigate the path explotion problem, we migrated to zesti.
        ZESTI identifies sensitive instructions(memory accesses and divisions.)
        dynamically while running the concrete input.  
        While sym exec it tries to exercise sensitive instructions on slightly
        different paths,  with the goal of triggering a bug if the respective
        instructions contain one.

      To mitigate the path explosion problem, ZESTI carefully chooses divergent
        paths via two mechanisms: (1) it only diverges close to sensitive
        instructions (memory accesses and divisions.), i.e instructions that
        might contain a bug, and (2) it chooses the divergence points in order
        of increasing distance from the sensitive instruction. 
      } 

    \end{itemize} 
}

\subsection{Debugger Logic for Pointer Analysis}
\frame
{
  \frametitle{\subsecname}
  \algoCheckerIII
    \cmt{
      Our work is on devising a debugger for static analysis.
      In our current implementation the static analysis that we are debugging is pointer analysis.
    } 
}

\subsection{Implicitly adding klee\_assumes}
\frame
{
  \frametitle{\subsecname}
  \lstII
    \cmt{
    Without the klee_assume, the dereference z->x may get resolved to many
      spurious memory objects. But while dealing with pointer analysis results
      we assume that the index of gptr is within bounds and as a result we are
      getting false positives that (The set of allocation sites corresponding
          to the memory object of the load address, z->x) is NOT a subset of
      (Points to allocation sites for the load address z->x) 
    }

}

\subsection{Importance of choosing a variable as symbolic}
\frame
{
  \frametitle{\subsecname}
  \lstI

}

\subsection{Which variables to make symbolic}
\frame
{
  \frametitle{\subsecname}
  \begin{itemize}
    \item Explicitly specifying which variables to make symbolic is difficult. 
    \vspace{1cm}
    \begin{itemize}
      \item Instrumented the code by inserting appropriate klee\_make\_symbolic.
      \item Rechability Analysis to figure out candidates to be made symbolic.
    \end{itemize}
  \end{itemize}
  \cmt{
    Rather that explicitly making the  inputs of the test program
      symbolic, let klee instrument the code by inserting klee_make_symbolic
      calls. The inputs that we are considering include command line arguments,
      file inputs, globals and variables used to read inputs (for example using
          scanf).
  }
}


\subsection{Bugs Found}
\frame
{
  \frametitle{\subsecname}
  \lstIII
}

\section{Questions?}
\subsection{Questions?}
\frame
{}

\end{document}
